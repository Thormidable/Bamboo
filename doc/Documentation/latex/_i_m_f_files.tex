IMF Files are generated by the IMF Compiler. The IMF Compiler is a seperate program to the DivWin engine and has a text based interface. The Compiler should be run from the terminal, so the user can view and use the interface.

Everytime the user runs the program will start with an empty IMF File. IMF Files contain media blocks. Each block begins with a type identifier to identify the type of media stored in the block. This is followed by a size specifier defining the amount of data in the remainder of the block. Finally the Block has a character string storing the reference.\par
 The main task the user will perform is to add Media to the IMF File. Each Media file added will require a reference (a character string which allows the media to be identified in the DivWin Engine) and often other data to fully define the object.\par
 Media can often be converted into several different types of IMF Blocks. e.g. A image can be converted to a 2D Texture, a Landscape height map or if it is 64 times taller than it is wide into a font. Each of these require different information to generate the object.\par
 An IMF File can contain many blocks all with different media types in. This allows the user to group media into sensible sets which are interdependant, eg a tank body model, a tank turret model, a tank shell model, a texture for the tank, and a model list representing the skeletal structure for the tank. This ensures that all inter dependant media can be loaded with a single call.\par
 Each level of the menu defines the options available to the user, selecting 0 will always move the user back up to the previous level. Otherwise, the user should select the desired option, insert the number representing it and press enter.\par


To add each item, select 1 from the main menu. Each file Name (including file type) which is entered will be added to the IMF File as a new block. The system will request all the information required to generate the object, then add it to the IMF File as a new block. Take care when selecting the references as they are the only way to access the media in the DivWin Engine.\par
 Once all the required media files are added to the IMF file, it can be written to the harddrive, by selecting option 7. The IMF file type should be included by the user.\par
 IMF Files can be loaded and will add all their blocks to the end of all the blocks in the current IMF File. This allows the user to add new media to previous groupings.\par


Media Types Supported:\par
 Shader Code:\par
 .shd (text files containing the shader code)\par
 Model Files:\par
 .x\par
 .obj\par
 .q3d\par
 Model Files can be converted into:\par
 Meshes (3D Models, including Normals and UV if available)\par
 Collision Objects (Currently only supporting convex faces)\par
 Box Collision Objects \par
 Images:\par
 .bmp\par
 Image Files can be converted into:\par
 2D Textures\par
 Fonts (are composed of 64 vertical characters, with equal width and height)\par
 Landscape Height Maps (Produces a map with a polygon per pixel in the image, with RGB(0,0,0) being no height and RGB(1,1,1) being maximum terrain height)\par
 \par
 User Generated Media Types:\par
 All data in these types is entered by the user into the IMF Compiler and the files are produced from that data. To activate them, insert the desired media type instead of a filename, in the add file menu.\par
 $\ast$ Shader Programs:\par
 Insert 'Shader Program' as file name.\par
 Model Lists:\par
 Insert 'Model List' as file name.\par
 Render Trees:\par
 Insert 'Render Tree' as file name.\par
 