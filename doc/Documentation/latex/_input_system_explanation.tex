There are two main types on input. Mouse inputs and Key inputs. Both are updated every frame and buffered until the next frame. This means that a key press or mouse position will be consistent throughout the entire frame. THe inputs re recieved as interupts so will be recieved in line with the OS. Key states are boolean and can be accessed using the macro \_\-KEY(Key Identifier). A list of Key Identifiers can be found on page \hyperlink{_key_identifiers_list}{III. List of Key Identifiers for accessing the Key States} KeyIdentifiersList. True is key pressed, false is key not pressed.\par
 \par
 The Mouse can be accessed using the macro \_\-MOUSE() which is a reference to the \hyperlink{classc_mouse}{cMouse} Object. The mouse has three buttons (for now) left, right and middle, because that is enough for most people. Like keys these are boolean values, with true for pressed and false for not pressed. The mouse also has x,y and z which are the cursor position in pixels from the window position 0,0. Finally the mouse has xs and ys which is the amount the cursor has moved (in pixels) since the last frame. 