\hyperlink{_w_t_div_win_8h_source}{WTDivWin.h}

Using DivWin by William Thorley

DivWin is based on a Process / Renderable Object mentality. Process' are classes with behavioural code which tell them how to behave. They take inputs from the OS, signal other process' and tell Renderable objects how to display. Once a Process is written, the user should create a new instance. The system will record the information add it to the system and make the process enact it's behaviour, with no further input form the user (Fire and Forget). This way many similar objects can be created easily once the behavioural code is written. Render Objects should not have any behavioural code and are entirely directed by Process'.

The Process' are controlled by \hyperlink{classc_kernel}{cKernel} which will track and update all the process'.

Render Objects actually display on the screen. They are moved around in 3D space and \hyperlink{classc_camera}{cCamera} renders them to the screen.

Collision Objects must be handed a Render Object at startup, which they will follow and determine collisions with other Collision Objects.

Files are loaded by the FileHandler and can be asked to provide media data for Render Objects and Collision Objects.

Lighting effects are controlled by the LightHandler.

The Main Loop:


\begin{DoxyCode}
*#include <WTDivWin.h>
*int WINAPI WinMain (HINSTANCE hInstance,
                    HINSTANCE hPrevInstance,
                    LPSTR lpCmdLine,
                    int iCmdShow)
*{
 //_START_PROGRAM(First_Process_Type,User_Settings_Type,Instance);  
 return _START_PROGRAM(cCore,cUserSettings,hInstance);

};
*
\end{DoxyCode}
 
\begin{DoxyCode}
 //Using DivWin in Linux
*#include <WTDivWin.h>

*int main ()
*{
  //_START_PROGRAM(First_Process_Type,User_Settings_Type); 
  return _START_PROGRAM(cCore,cUserSettings);

*}
*
\end{DoxyCode}


Main Loop Explanation: This will call the main function which will initialise the various components of the system. First\_\-Process\_\-Type must be a class type which inherits \hyperlink{classc_process}{cProcess} and should be the process that initialises and creates all the other processes required for the game. User\_\-Settings\_\-Type must be a class type which inherits \hyperlink{classc_settings}{cSettings}. The virtual function UserSettings() in the derived class should set all the variables the user wishes to define. If the User does not wish to define ANY settings, use \hyperlink{classc_settings}{cSettings} here and the defaults will be used. hInstance is a Windows only variable which is passed from the operating system to the WinMain function. Use the first HINSTANCE Passed to the function.\par
 First \_\-START\_\-PROGRAM will set the settings for the game to use.\par
 Then it will initialise the various components of the engine.\par
 It will create an instance of the class type First\_\-Process\_\-Type which should initilise the game and create other processes to start the game.\par
 While there are processes alive (and the system has not received an exit signal) the system will continue to cycle.\par
 As the system exit, it will clear data, close links to devices and hardware and return a suitable exit signal for the Operating system. This should be returned from the funciton main().\par
 