\hyperlink{_w_tc_process_8h_source}{WTcProcess.h}

Using William Thorley’s Process system

The First Process is exactly the same as every other process. i.e. it must inherit \hyperlink{classc_process}{cProcess}. What is special about the Process passed to the \_\-START\_\-PROGRAM() call is that it is the automatically called when the Kernel is initialized. This means that the loading code for cCore is the initialization code for the entire program. Usually the process’ which will form the program are started here as well as loading files for the system. For these examples I wil call the class cCore.

This shows the declaration for cCore. Using \_\-PROCESS(Type) is the same as calling class Type : public \hyperlink{classc_process}{cProcess} to inherit \hyperlink{classc_process}{cProcess}. Declaration: 
\begin{DoxyCode}
 _PROCESS(cCore)
 {
  public:
  cCore();
 
  void Run();
 
  ~cCore();
 
 void AdditionalKillFunctionality();
 void AdditionalSleepFunctionality();
 void AdditionalWakeFunctionality();
 
 };
\end{DoxyCode}


cCore(): This shows the constructor for cCore. As it is the first object it will load media for the rest of the program and kick off the other processes. 
\begin{DoxyCode}
 cCore::cCore()
 {
 //Load IMF Files into memory.
  _LOAD_FILE("../src/User/Models/AShipModel.imf");
  _LOAD_FILE("../src/User/Models/StartShip.imf");
  _LOAD_FILE("../src/User/Textures/ATexture.imf");
  _LOAD_FILE("../src/User/CollisionObjects/ACollision.imf");
  _LOAD_FILE("../src/User/Objects/AIMFFileWithSeveralComponents.imf");

 // Setup the camera.
  _CAMERA->Far(1000.0f);
  _CAMERA->Frustum();
 
 //Create a new process and a renderable object.
  mpProcessPointer=_CREATE(MyFirstProcess());
  mpRenderObjectPointer=_CREATE(MyFirstRenderObject());

 }
\end{DoxyCode}


Run(): Run() is run once every frame as long as cCore is alive and controls the entire program. The Run() function should is explicity designed to be re-\/written by the user to give each \hyperlink{classc_process}{cProcess} class their functionality. This Code can be seen as the behaviour that the process should follow. It should rotate and position the Process' Renderobjects, send signals, receive inputs, anything the Process may want to do as part of it's behaviour. The function cCore::Run() can be used like any other process, but is usually used to oversee the running of the program. 
\begin{DoxyCode}
 void cCore::Run()
 {

  if(_KEY(KEY_SPACE)) _CAMERA->Far(10.0f);
   else _CAMERA->Far(1000.0f);
  _CAMERA->Frustum();

  mpProcessPointer=_CREATE(MyFirstProcess());
  mpRenderObjectPointer=_CREATE(MYFirstRenderObject());

 }
\end{DoxyCode}


$\sim$cCore(): This is called when the object is deleted, not when it is killed. It is difficult to predict reliably where this will be called, so it is best not to put code in this function.

AdditionalKillFunctionality(): This is called whenever the cCore Process is Killed. This will only activate if the process was alive and is now dead. This should be used to kill or transfer control of Render Objects that are owned by this process, or unload files that are no longer used.


\begin{DoxyCode}
 *void cCore::AdditionalKillFunctionality()
 *{
 _KILL(mpRenderObjectPointer);
 mpProcessPointer_SIGNAL(_S_SLEEP);
         *}
         *
\end{DoxyCode}


AdditionalSleepFunctionality(): This is called whenever the cCore Process is made to Sleep. This will only activate if the process was awake and is now asleep. This is generally used to sleep Render Objects that are owned by this process.


\begin{DoxyCode}
         *void cCore::AdditionalSleepFunctionality()
         *{
 mpRenderObjectPointer->Signal(_S_SLEEP);
                 *}
                 *
\end{DoxyCode}


AdditionalWakeFunctionality(): This is called whenever the cCore Process is made to Wake. This will only activate if the process was asleep and is now awake. This is generally used to wake Render Objects that were slept when cCore was sent to sleep.

void cCore::AdditionalSleepFunctionality() \{ mpRenderObjectPointer-\/$>$Signal(\_\-S\_\-WAKE); \}  